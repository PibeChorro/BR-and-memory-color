\documentclass[]{article}
\usepackage[citestyle=authoryear,backend=bibtex]{biblatex}
\addbibresource{LibraryVincent.bib}

%opening
\title{Binocular Rivalry in naturalistic scenes}
\author{Vincent Plikat}

\begin{document}

\maketitle

\begin{abstract}

\end{abstract}

\section*{Introduction}
It is widely accepted that perception relies on expectations, predictions and prior knowledge in order to facilitate the interpretation of ambiguous and noisy sensory input \parencite{clarkWhateverNextPredictive2013}. 
%One can image to see a vague silhouette of a creature waling on all fours in a distance on a snowy landscape. Probably everyone would interpret it as a polar bear, whereas seeing the same vague silhouette in a desert would be interpreted as lion.
The influence of expectations is apparent during the perception of ambiguous stimuli, in which the same visual input has two mutually exclusive interpretations. Here prior expectations modulates which interpretation is consciously perceived \parencite{panichelloPredictiveFeedbackConscious2012}. 
However, these ambiguous figures all show one coherent image that can be interpreted in different ways, the story becomes much more complicated in situations where two mutually exclusive visual signals need to be processed. This mostly artificially generated situation is called binocular rivalry (BR) in which one eye is presented with a stimulus (e.g., a face) and the other with a mutually exclusive stimulus (e.g., a house). Here the conscious percept predominantly alternates between both interpretations instead of seeing a mixture of both (although also apparent and called piece-mealing). BR has been investigated for centuries and many factors that drive a stimulus to be dominant were found, such as luminance, contrast \parencite{brascampLawsBinocularRivalry2015}, but also higher level factors such as attention, aesthetic or monetary value and imagery \parencite{safaviMultistabilityPerceptualValue2022}. 
Two key properties of BR namely onset rivalry - which stimulus is perceived first, and sustained rivalry - which stimulus is perceived longer over a certain period of time, have been analyzed and explained in a predictive coding framework \parencite{hohwyPredictiveCodingExplains2008}. It states that both are driven by the likelihood and prior probability of each stimulus and the prediction error that results from suppressing one stimulus.
Even though this view is quite influential very few studies have systematically manipulated the predictability of stimuli. The ones that do, show conflicting results. It has been shown that showing rotating gratings presented to both eyes biases the perception of rivaling gratings towards the expected direction in onset rivalry \parencite{denisonPredictiveContextInfluences2011, attarhaOnsetRivalryFactors2015}. Showing structured and random series of motion, shape and idioms in a rivaling condition also reveals a bias towards the predicted stimulus \parencite{huVisualTemporalIntegration2021}.
However, presenting scenes with congruent and incongruent objects in a rivaling condition, shows a bias in sustained dominance towards the scene containing an incongruent (thus unpredicted) object \parencite{mudrikSceneCongruencyBiases2011,zachariaDoesValenceInfluence2020} and showing an unpredicted image in a learned sequence using naturalistic stimuli shows a bias prioritizing the surprising stimulus at rivalry onset \parencite{denisonPerceptualSuppressionPredicted2016}.
These contradictory results seem hard to integrate at first, however inspecting the methods in more detail, gives room for a plausible explanation.
The studies showing evidence for dominance of the expected stimulus used artificial and simple stimuli and induced predictions through time series \parencite{denisonPredictiveContextInfluences2011,attarhaOnsetRivalryFactors2015,huVisualTemporalIntegration2021}. Studies showing evidence for dominance of the unexpected stimuli used naturalistic and complex stimuli and induced predictability through static scene context \parencite{mudrikSceneCongruencyBiases2011,zachariaDoesValenceInfluence2020}. 
Given that many different factors can modulate dominance during BR from low \parencite{brascampLawsBinocularRivalry2015} to high-level features \parencite{safaviMultistabilityPerceptualValue2022} and a hierarchy in consciousness \parencite{zekiDisunityConsciousness2003}, it is reasonable to believe that predictions about sensory input on different levels of stimulus complexity influence conscious perception differently.\par
Predictions based on stimulus properties over time should dominate at rivalry onset. On a neuro level temporal predictions and low level stimulus features work through lateral connections in low-level areas (V1 etc.)
Violations of predictions based on semantic scene context should induce a strong prediction error, leading to fast switches switch proportional to strength of prior violated.
On a neuro level contextual semantic predictions come from higher areas and receive prediction errors that render the current dominant percept instable
\begin{itemize}
	\item Perception relies on prior knowledge, expectation, prediction
	\item In ambiguous figures (e.g., Necker cube, duck-rabbit, old-young lady) exogenous factors (priming, contextual information) and endogenous factors (attention, imagery) can modulate the percept
	\item these present a single coherent sensory input
	\item BR is different - it shows two conflicting images 
	\item influence of exogenous factors on BR investigated - e.g., contrast, spatial frequency, luminance, complexity
	\item influence of endogenous factors investigated - e.g., attention, imagery, preference, valence
	\item influence of predictability are rare and ambiguous!
\end{itemize}
Studies showing expected/unexpected stimuli in binocular rivalry 
\begin{itemize}
	\item Rotating gratings - expected stimulus onset dominant and longer initial perception duration \cite{denisonPredictiveContextInfluences2011,attarhaOnsetRivalryFactors2015,huVisualTemporalIntegration2021}
	\item structured sequences (shape and idioms) \cite{huVisualTemporalIntegration2021}.
	Motion and shape benefit from unconscious processing. Idioms benefit from conscious processing
	\item In/congruent objects in scenes are dominant \cite{mudrikSceneCongruencyBiases2011,zachariaDoesValenceInfluence2020}
	\item Natural images (hence higher prior) dominate over unnatural (lower prior) images \cite{bakerNaturalImagesDominate2009}
\end{itemize}
Studies using CFS with predicted/unpredicted stimuli
\begin{itemize}
	\item expected motion and shape break through faster \cite{huVisualTemporalIntegration2021}
	\item primed words and targets with shared parts of the prime (e.g., rock and roll) break through faster \cite{costelloSemanticSubwordPriming2009}
	\item incongruent objects break through faster than congruent \cite{mudrikIntegrationAwarenessExpanding2011}
\end{itemize}

\printbibliography
\end{document}
